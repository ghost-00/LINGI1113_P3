\documentclass[10pt, onecolumn] {IEEEtran}

\usepackage[utf8x]{inputenc}
\usepackage[frenchb]{babel}
\usepackage[T1]{fontenc}
\usepackage{lmodern}
\usepackage{fullpage}
\usepackage{graphicx}
\usepackage{epstopdf}
\usepackage{caption}
\usepackage{subcaption}
\usepackage{multirow}
\usepackage{hyperref}

% Math symbols
\usepackage{amsmath}
\usepackage{amssymb}
\usepackage{amsthm}
\usepackage{mathtools}

% Numbers and units
\usepackage[squaren, Gray]{SIunits}
\usepackage{sistyle}
\usepackage[autolanguage]{numprint}
%\usepackage{numprint}
\newcommand\si[2]{\numprint[#2]{#1}}
\newcommand\np[1]{\numprint{#1}}

\DeclareMathOperator{\newdiff}{d} % use \dif instead
\newcommand{\dif}{\newdiff\!}
\newcommand{\fpart}[2]{\frac{\partial #1}{\partial #2}}
\newcommand{\ffpart}[2]{\frac{\partial^2 #1}{\partial #2^2}}
\newcommand{\fdpart}[3]{\frac{\partial^2 #1}{\partial #2\partial #3}}
\newcommand{\fdif}[2]{\frac{\dif #1}{\dif #2}}
\newcommand{\ffdif}[2]{\frac{\dif^2 #1}{\dif #2^2}}
\newcommand{\constant}{\ensuremath{\mathrm{cst}}}

\newcommand{\bigoh}{\mathcal{O}}

% cfr http://en.wikibooks.org/wiki/LaTeX/Colors
\usepackage{color}
\usepackage[usenames,dvipsnames,svgnames,table]{xcolor}
\definecolor{dkgreen}{rgb}{0.25,0.7,0.35}
\definecolor{dkred}{rgb}{0.7,0,0}

\usepackage{listings}
\lstset{
  numbers=left,
  numberstyle=\tiny\color{gray},
  basicstyle=\rm\small\ttfamily,
  keywordstyle=\bfseries\color{dkred},
  frame=single,
  commentstyle=\color{gray}=small,
  stringstyle=\color{dkgreen},
  %backgroundcolor=\color{gray!10},
  %tabsize=2,
  rulecolor=\color{black!30},
  %title=\lstname,
  breaklines=true,
  framextopmargin=2pt,
  framexbottommargin=2pt,
  extendedchars=true,
  inputencoding=utf8x
}

\lstset{language={C}}

\title{INGI1113\\
Minix3 Projet 2013-2014\\
Rapport : Groupe 23}

\author{Jos Zigabe  \and Mehdi Dumoulin}

\begin{document}

\maketitle
\tableofcontents
\newpage

\section{introduction au projet}

Dans le cadre du cours de \og Systèmes informatiques 2 \fg, il nous a été demandé pour le troisième projet de modifier l'OS Minix3. Dans un premier temps afin de se familiariser avec l'architecture de Minix, on a dû implémenter un appel système \texttt{get\_pidinfo} à ajouter au code source et ensuite on a dû implémenter un nouveau mode d'accès des fichiers système pour MFS qui permettra de les crypter.      

\subsection{Description de notre architecture}

Pour la première phase, on donc dû implémenter l'appel système \texttt{getpidinfo} grâce à la fonction : \texttt{get\_pidinfo(struct pid\_s *pids)} avec en paramètre une structure censée contenir l'ID du processus parent et de sont fils. La fonctionnalité de cette fonction est très simple elle doit juste récupérer les ID du parent et du fils, en lançant un appel système à  \texttt{getpidinfo}.\\

\subsubsection{Descriptions des fichiers ajoutés et modifiés }

Afin de rajouter notre appel nous sommes passé par deux étapes : la première consistant à créer un gestionnaire d'appel-système qui est la fonction appelée en réponse à l'appel de \texttt{getpidinfo} et une librairie utilisateur qui assemble les paramètres pour l'appel système et qui appel le gestionnaire de notre appel  \texttt{getpidinfo}.\\

\begin{itemize}
\item \textbf{Gestionnaire de système d'appel} : Premièrement il nous a fallu localiser le serveur adéquat qui dans notre cas est pm \og process manager \fg qui s'occupe des appels systèmes relatifs à la gestion de processus ou encore à examiner les propriétés des processus ce qui est notre cas. Ensuite, chaque serveur possède 2 fichiers : \\ \texttt{table.c} qui contient la définition d'un tableau de pointeurs de fonction indexé par le numéro du système d'appel avec donc à chaque ligne l'adresse de la fonction du gestionnaire d'appel. Nous avons donc dû rajouter dans ce tableau \texttt{call\_vec} la fonction du gestionnaire d'appel de \texttt{do\_getpidinfo,	\slash* 31 = pidinfo	*\slash} indexé au numéro de son appel système. Il y a ensuite le fichier  \texttt{proto.h} qui contient les prototypes de toutes les fonctions des gestionnaires d'appel système, dans lequel nous avons dû également y inclure notre prototype \texttt{\_PROTOTYPE( int do\_getpidinfo, (void)					)}. Et on a également implémenté la fonction  \texttt{do\_getpidinfo} dans le fichier /src/servers/pm/misc.c.\\

\item \textbf{Fonction de librairie d'utilisateur} : Tout d'abord, nous avons dû modifier le fichier /src/include/minix/callnr.h afin d'attribuer un numéro à l'appel système de la fonction du gestionnaire \texttt{\#define GETPIDINFO 31}. Ensuite nous avons implémenté la fonction de librairie d'utilisateur \texttt{\_getpidinfo.c} dans le dossier /src/lib/libc/posix/ afin que l'appel puisse être utilisé par n'importe quel programme C.  \\
\end{itemize}

Pour finir nous avons implémenté la structure \texttt{pid\_s} que nous avons placé dans le fichier /src/include/sys/pid\_s.h, fichier dans lequel on a mis le prototype de la fonction : \texttt{\_PROTOTYPE( int getpidinfo, (struct pid\_s *pids)            )}. Ensuite, nous avons ajouté le fichier /src/lib/libc/syscall/getpidinfo.S qui permet de convertir \texttt{getpidinfo} en  \texttt{\_getpidinfo} afin que l'appel à \texttt{getpidinfo(struct pid\_s *pids);} soit possible. 

\newpage
\section{Projet principal}

Le but du projet Minix de cette année était de faire un système transparent de cryptage/décryptage de fichier qui s'active lorsque l'on monte une partition avec une certaine option. Nous allons discuter ici de nos choix d'implémentations, de l'architecture globale de notre projet et des limites de notre programme.\\   

\subsection{Architecture}

Avant d'implémenter notre solution, nous avions tout d'abord compris que le \texttt{VFS} était une sorte de système de fichier abstrait qui était au-dessus des autres \texttt{FS}. Il offre ainsi une interface d'accès commune à tous les autres systèmes de fichier qui cache de ce fait les propriétés de chaque \texttt{FS} dans le \texttt{VFS}. 

\begin{figure}[h]
\begin{center}
\includegraphics [scale=0.5] {figures/VFS-FS.png} 
\caption{Hiérarchie entre \texttt{FS}}
\label{default}
\end{center}
\end{figure}
     
Le \texttt{VFS} accède à l'implémentation des \texttt{FS} à travers des requêtes vers des fonctions et communique avec eux grâce aux \texttt{messages}. \\

Pour que notre système de cryptage/décryptage fonctionne avec tous les \texttt{FS}, nous avons décidé de l'implémenter dans \texttt{VFS}. En effet, nous avons mis notre code dans le fichier \texttt{read.c} de \texttt{VFS} dans la méthode \texttt{read\_write}. Cette méthode fait appel à une requête pour pouvoir appeler le  \texttt{FS} correspondant au disque monté et lire la partie du fichier voulue. Cette méthode permettant à la fois de lire et d'écrire dans un fichier, le cryptage/décryptage se situe à différents endroits :

\begin{itemize}
\item Lecture dans un fichier : ici il faut intercepter le buffer, après que le \texttt{FS} ai fait son boulot, grâce à \texttt{sys\_datacopy}, ensuite il suffit de le décrypter et de renvoyer cette nouvelle version au processus utilisateur.
\item Ecriture dans un fichier : dans ce cas-ci, il faut intercepter le buffer qui est passé au \texttt{FS} et qui contient le contenu de ce que l'on veut écrire dans le fichier pour pouvoir le crypter. Pour ce faire, on utilise la fonction \texttt{sys\_datacopy}, qui permet de récupérer les données contenues dans l'espace utilisateur, et on applique notre méthode de cryptage sur ce contenu. Nous discuterons plus loin de la méthode utilisée pour le cryptage/décryptage.
\end{itemize}

\subsection{Commande mount}

Nous avons du modifier le fichier \texttt{mount} de minix, qui permet de monter une partition, afin qu'il prenne en compte un nouveau paramètre qui est le \texttt{-e}. Si le \texttt{-e} est mis lorsqu'une partition est montée ça avertira Minix que les fichiers doivent être crypté/décrypté.
Nous avons fait une copie du fichier \texttt{mount} original et nous l'avons mis dans le répertoire \texttt{/test} de Minix en le modifiant au préalable.
Pour stocker le fait qu'une partition soit montée en mode crypté nous avons utilisé la même technique que le \texttt{readonly}, à savoir nous faisons un \texttt{OR Logique} sur la variable \texttt{mountflags} de \texttt{mount.c} avec une nouvelle constante créée pour l'occasion qui s'appelle \texttt{MS\_CRYPT} et qui est déclaré dans le fichier \texttt{include/sys/mount.h}. Nous lui donnons comme valeur \texttt{0x008} en hexadécimal.
Le fait de faire un \texttt{OR Logique} et de lui donner une telle valeur permet à la variable \texttt{mountflags} de comporter plusieurs informations. En effet, elle peut à la fois savoir si la partition est en \texttt{readonly} et si la partition est cryptée ou non. Pour retrouver ces valeurs, il suffit de faire un \texttt{ET Logique} dessus avec les variables définies dans \texttt{include/sys/mount.h}

\subsection{Gestion de la clé}

Nous avons choisi de demander la clé à l'utilisateur et non d'enregistrer la clé dans le \texttt{superblock} car la modification de ce dernier est fort complexe. Ainsi dans la fonction \texttt{mount} du fichier \texttt{/src/lib/libc/other/\_mount.c} on teste d'abord le \texttt{mountflag} et si la valeur contenue est \texttt{MS\_CRYPT}, on alloue de l'espace pour la clé de cryptage que l'on demande via un \texttt{scanf}. Ensuite, on ne savait pas trop comment intégrer cette clé dans le \texttt{message} afin que le \texttt{VFS} puisse l'utiliser, étant donnée qu'il n'y avait plus de sous-variable dans le structure \texttt{message} pour la contenir. Face à ce problème nous avons donc décidé d'ajouter une sous-variable \texttt{char *m1p4} de la structure \texttt{mess\_1} définie dans le fichier \texttt{/src/include/minix/ipc.h}. Et suite à cette modification de la structure \texttt{mess\_1}, on a pu transférer la clé provenant de l'espace utilisateur au \texttt{VFS}. Enfin étant donné que l'on était incapable de manipuler le \texttt{superblock}, on a décidé pour la suite d'enregistrer la clé dans la structure \texttt{vmnt} qui stocke les informations relatives à la partition attachée. Ainsi nous avons modifié cette structure en ajoutant une nouvelle sous-variable afin de contenir la clé du cryptage. 

\subsection{Encodage dans les fichiers}

La méthode de cryptage/décryptage fournie utilise des blocs de 8 bytes, nous avons donc dû tenir compte de cette restriction lors de l'implémentation de notre programme. Le contenu que l'on souhaite écrire dans un fichier, qu'il soit crypté ou non, est rarement un multiple de 8. Par conséquent, nous devons compléter le contenu avec des caractères quelconques afin de pouvoir le crypter, mais en faisant cette opération nous devons aussi nous souvenir de combien de caractères nous avons du ajouter. Nous avons décidé de pallier ce problème en écrivant dans tous les fichiers cryptés un bloc de 8 bytes contenant le nombre de caractères qui restait dans le dernier bloc de 8 bytes avant de devoir le compléter. Ce bloc sera inséré au début du fichier crypté dès le début de l'écriture. Lorsqu'on écrira quelque chose à la suite d'un fichier, ce bloc sera mis à jour. Lors de la lecture d'un fichier crypté, ce bloc est la première chose lue pour pouvoir déterminer combien de caractères doivent être retirés à la fin.
L'écriture ou la lecture dans un fichier crypté est alors assez triviale, il suffit de diviser le contenu à écrire/lire en bloc de 8 et le crypter/décrypter en complétant éventuellement le dernier bloc si celui-ci n'est pas complet.

\subsection{Cryptage}

Nous avons utilisé l'interface fournie sur foditic pour réaliser l'encryption TEA. Elle prend comme paramètres une clé et deux chaines de caractères de 8 octets.

\subsection{Limite de notre programme}

Notre programme, de par son implémentation, présente certaines limites d'utilisation, nous allons les détailler ici :
\subsubsection{Ecriture en clair dans un fichier crypté}

Notre implémentation nous impose d'écrire un bloc de 8 bytes au début d'un fichier crypté pour savoir combien de bytes ont été ajoutés à la fin pour arriver à un multiple de 8. Du fait de ces 8 bytes, notre implémentation ne nous permet pas de pouvoir écrire à la suite d'un fichier crypté en clair ou d'écrire du contenu crypté à la suite d'un fichier contenant déjà du texte non crypté, car dans ces cas ce bloc de 8 bytes ne représente plus la réalité ou ne sait pas être écrit au début du fichier.
Une piste de solution pourrait être que chaque bloc de 8 bytes écrit dans un fichier (en clair ou crypté) comportent le nombre de bytes qui ont du éventuellement être rajouté à la fin pour arriver à 8 bytes et si ce nombre vaut 0 le fichier est en clair et s'il vaut 9 le fichier est crypté par exemple. Néanmoins, cette méthode rendrait le fichier tel qu'il est physiquement sur le disque assez différent du fichier tel qu'il est affiché sur l'écran et augmenterait considérablement la taille des fichiers.
\subsubsection{Certains problèmes d'affichage}

Pour des raisons assez obscures, Minix augmente la taille des fichiers de temps en temps. De fait, lorsque l'on affiche sur l'écran le contenu d'un fichier Minix s'attend à afficher beaucoup plus de caractères que prévu et donc certains caractères indésirables apparaissent à la fin du contenu. Ce problème est plutôt aléatoire et nous n'avons pas réussi à lui trouver une cause, nous avons tenté d'insérer dans le premier bloc de 8 bytes une champ avec la taille du fichier, mais étant donné que la méthode \texttt{read\_write} est appelée plusieurs fois pour une même lecture/écriture cette valeur ne représentait plus la taille réelle du fichier.
À côté de cela, nous avons de temps en temps certains malloc qui provoque un kernel panic dans le VFS, pareil pour les free qui provoquent des kernel panic tout le temps.
L'utilisation de la commande \texttt{vi} pose aussi quelques souci. En effet, il affiche correctement le contenu du fichier en clair, mais rajoute des caractères incohérents à la fin.

\subsubsection{Fichiers test}

Un petit fichier test est founir avec un Makefile, il suffit de taper Make et il se charge de monter les disques et d'écrire dans les fichiers + les lire par la suite.

\end{document}